\section{Auswertung}
\label{sec:Auswertung}
Die Grafiken und Berechnungen, die in Abschnitt \autoref{sec:Auswertung} gezeigt werden, wurden unter Verwendung der Python-Bibliotheken Matplotlib \cite{matplotlib}, Scipy \cite{scipy} und Numpy \cite{numpy} erstellt. 
Zur Berücksichtigung von Unsicherheiten wurde Uncertainties \cite{uncertainties} verwendet.

\subsubsection{}
%Tabellen nebeneinander

\begin{table}[htbp]
  \centering
  \caption{Caption for the table}
  \begin{minipage}[t]{0.45\linewidth}
    \centering
    \begin{tabular}{c|c}
      \hline
      \textbf{Header 1} & \textbf{Header 2} \\
      \hline
      -16 & 19\\
      -15 & 60\\
      -14 & 129\\
      -13 & 254\\
      -12 & 369\\
      -11 & 467\\
      -10 & 529\\
      -9 & 631\\
      -8 & 756\\
      -7 & 850\\
      -6 & 901\\
      -5 & 969\\
      -4 & 968\\
      -3 & 921\\
      -2 & 958\\
      -1 & 933\\
      \hline
    \end{tabular}
  \end{minipage}
  \begin{minipage}[t]{0.45\linewidth}
    \centering
    \begin{tabular}{c|c}
      \hline
      \textbf{Header 1} & \textbf{Header 2} \\
      \hline
      0 & 918\\
      1 & 826\\
      2 & 759\\
      3 & 584\\
      4 & 468\\
      5 & 330\\
      4 & 468\\
      5 & 330\\
      6 & 308\\
      7 & 244\\
      8 & 196\\
      9 & 137\\
      10 & 73\\
      11 & 38\\
      12 & 8\\
      \hline
    \end{tabular}
  \end{minipage}
\end{table}

%Plots und Bilder
%\begin{figure}[H]
%  \includegraphics[width=\linewidth]{plots/.pdf}
%  \caption{}
%  \label{fig:}
%\end{figure}

\newpage