\section{Zielsetzung}
\label{sec:Zielsetzung}
Ziel des Versuches ist es, die mittlere Lebensdauer von kosmischen Myonen mittels eines Szintillationsdetektors zu bestimmen. 

\section{Theorie}
\label{sec:Theorie}
\subsection{Eigenschaften und Entstehung kosmischer Myonen}
Kosmische Myonen entstehen in einer höhe von ca. $\SI{20}{km}$ in sogenannten Luftschauern. Diese finden statt, wenn ein hochenergetisches Teilchen aus dem kosmos auf die Erdatmosphäre trifft und dort mit den Teilchen dieser wechselwirkt.
Bei nicht zu hohen Energien sind Protonen die häufigsten Primärteilchen. Diese Treffen auf die Atmosphäre und zerfallen bei der Wechselwirkung mit den Atomen und Molekülen der Erdatmosphäre in Kaonen und Pionen. Kaonen und Pionen zerfallen weiter in Myonen welche als häufigstes Teilchen auf Meereshöhe gemessen werden.
\\
\\
Bild von Schauer
\\
\\
Grund für das erreichen des Bodens der Myonen sind relativistische Effekte. Ein Myon mit einer Energie von $E_{\mu} = \SI{10}{GeV}$ fliegt nach 
\begin{equation}
    \label{eqn:t1}
    v = \frac{e}{e}
\end{equation}

\subsection{Zerfallsgesetz und mittlere Lebensdauer}
\subsection{Szintillationsdetektor}


\cite{sample}
\cite{astro}

\newpage
