\section{Durchführung}
\label{sec:Durchführung}
Um die Lebensdauer von Myonen korrekt zu Messen müssen einige Kallibrationsschritte vorgenommen werden. Zunächst verbindet man die bereits fest verbauten Photomultiplier jeweils mit einer variablen Verzögerungsleitung.
Diese werden jeweils mit einem Diskriminator verbunden und an ein Oszilloskop angeschlossen. An dem Oszilloskop kann die Signalbreite abgelesen werden, welche durch das drehen einer Stellschraube an den Diskriminatoren verstellt werden kann. 
Die Signalbreite wird auf $\SI{10}{ns}$ eingestellt um einen ausreichenden Überlapp der Signale in der Koinzidenzschaltung zu gewehrleisten. Ist die Signalbreite korrekt eingestellt, werden die ausgänge der Diskriminatoren an die Impulszähler angeschlossen. Der Threshhold an den Diskriminator wird dann jeweils so variiert, dass eine Zählrate von $\SI{30}{Impulse\per\second}$ gemessen wird. 
Die kalibrierten Signale der Diskriminatoren werden in die Koinzidenzschaltung gespeist, und eine Messreihe wird aufgenommen, bei welcher für verschiedene Verzögerungen in beiden Verzögerungsleitungen die Impulsrate am Ausgang der Koinzidenzschaltung gemessen wird. Die Messdaten werden in einem Graphen aufgetragen und eine passende Verzögerung wird gewählt, sodass die Impulsrate am Ausgang der Koinzidenzschaltung stabil maximal ist. 
Anschließend wird der Rest der Schaltung aufgebaut und passende Werte für die Suchzeit am Monoflop und die maximale Zeitspanne am TAC eingestellt. Um die Channel zu kalibrieren wird der Ausgang der Koinzidenzschaltung durch einen Doppelimpulsgenerator ersetzt, der Impulse mit
einem bekannten Intervall zwischen den Pulsen generiert. So kann am Computer der Channel für die eingestellten Impulsdauern abgelesen werden und eine Messreihe wird aufgenommen. Nach den Kallibrationsschritten wird die restliche Schaltung wie in \autoref{sec:aufbau} verkabelt und der Messprozess an den Impulszählern wird zeitgleich mit dem Messprozess am PC gestartet.

\newpage