\section{Diskussion}
\label{sec:Diskussion}
Im Folgenden werden die prozentualen Abweichungen nach
\begin{equation}\label{eq:1}
    \Delta = |\frac{exp - theo}{theo}|\cdot 100\%
\end{equation}
berechnet.

%Tabelle
%\begin{table}[H]
%    \centering
%    \caption{Experimentell bestimmte Halbwärtszeit $T_{exp}$, Theoriewert $T_{theo}$ und Abweichung in \%.}
%    \begin{tabular}{c c c}
%        \toprule
%        {$T_{exp}\,/\symup{s}$} & {$T_{theo}\,/\symup{s}$} & {$\Delta\,/\symup{\%}$}\\
%        \midrule
%        
%        \bottomrule
%    \end{tabular}
%    \label{tab:Diskussion}
%\end{table}

\subsection{Stabilitätsbedingung}
Für die Überprüfung der Stabilitätsbedingung konnte für beide Resonatoren nicht die in \autoref{sec:a1} berechnete theoretische obere Resonatorlänge vermessen werden. Grund dafür ist, dass bei dem bikonkaven Resonator die optische Schiene keine ausreichende länge hat um $\SI{2800}{mm}$ zu erreichen. Bei dem plan konkaven Aufbau war die LASER justierung sehr aufwändig, weswegen nicht ausreichend Zeit zur verfügung stand um bis zu einer Resonatorlänge von $\SI{1400}{mm}$ zu erreichen.
Trotzdessen ist in den Daten zu erkennen, dass in den theoretisch stabilen Bereichen der LASER stabil funktioniert und die Leistung des LASERs unabhängig ist von der Resonatorlänge. 
\subsection{Wellenlänge des LASERs}
\begin{table}[H]
    \centering
    \caption{Experimentell bestimmte Halbwärtszeit $T_{exp}$, Theoriewert $T_{theo}$ und Abweichung in \%.}
    \begin{tabular}{c c c}
        \toprule
        {$T_{exp}\,/\symup{s}$} & {$T_{theo}\,/\symup{s}$} & {$\Delta\,/\symup{\%}$}\\
        \midrule
        
        \bottomrule
    \end{tabular}
    \label{tab:Diskussion}
\end{table}
\subsection{Polarisationsrichtung des LASERs}
\subsection{Longitudinale Moden des LASERs}
\subsection{Transversale Moden des LASERs}
\newpage