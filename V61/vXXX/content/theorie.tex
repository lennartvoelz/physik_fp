\section{Zielsetzung}
\label{sec:Zielsetzung}
Ziel ist es die Funktionsweise eines Helium-Neon-Lasers zu verstehen. Der Laser wird dazu aufgebaut und hinsichtlich seiner Eigenschaften, wie Wellenlänge, Intensitätsverteilung
senkrecht zur Ausbreitungsrichtung, Polarisation und Modenstruktur, untersucht.

\section{Theorie}
\label{sec:Theorie}
Ein Laser (Light Amplification by Stimulated Emission of Radiation) erzeugt kohärentes, nahezu monochromatisches und gerichtetes Licht. Es gibt verschiedene Wege einen Laser
zu realisieren, der hier verwendete Helium-Neon-Laser gehört zu den Gaslasern.\\
Ein solcher Laser lässt sich grundlegend in drei Teile unterteilen: ein aktives Medium, eine Pumpquelle und einen Resonator.\\
\\
Das aktive Medium ist in diesem Versuch ein Gasgemisch aus Helium und Neon. Diese Wahl ist begründet durch die Tatsache, dass Helium und Neon angeregte Niveaus besitzen, die
nahe beieinander liegen. So kann das Helium durch die Pumpquelle angeregt werden und die Energie auf das Neon übertragen. Der Energiezustand der übertragenen Energie ist jedoch
nicht das niedrigste angeregte Niveau, sondern ein metastabiler Zustand, wodurch eine Besetzungsinversion länger erhalten bleibt. Der Übergang in diesen metastabilen Zustand
sorgt für die Emission des charakteristischen roten Lichts. Dieser Prozess geschieht entweder spontan oder durch stimulierte Emission. Bei der spontanen Emission wird ein Photon
emittiert, wenn ein Elektron von einem angeregten Zustand in den Grundzustand zurückfällt. Bei der stimulierte Emission wird ein Photon emittiert, wenn ein einfallendes Photon
das Elektron von einem angeregten Zustand in den Grundzustand zurückfallen lässt. Dieses emittierte Photon hat dabei die gleiche Energie, Phase und Ausbreitungsrichtung wie das
einfallende Photon.\\
\\
Die Pumpen ist in diesem Versuch ein elektrischer Entladungsprozess. Dabei wird das Helium durch eine angelegte Spannung ionisiert und in einen angeregten Zustand versetzt.
Durch die Energiepumpe wird im Lasermedium eine vom thermischen Gleichgewicht abweichende Besetzung eines Niveaus erzeugt. Mit ausreichender Pumpenergie kann so eine Besetzungsinversion (angeregter Zustand höhere Besetzungszahl als Grundzustand) erreicht werden.\\
\\
Der Resonator besteht aus zwei Spiegeln, von denen einer teilweise durchlässig ist. Senkrecht auf die Spiegel treffende Photonen verbleiben im Resonator und sorgen für 
eine Verstärkung des Lichts durch stimulierte Emission. Diese Verstärkung ist dabei umso größer, je länger die Verweildauer des Photons im Resonator ist. Die Verweildauer ist
Die Verweildauer ist dabei abhängig von der Länge des Resonators und der Reflektivität der Spiegel. Dadurch, dass ein offener Resonator verwendet wird, entsteht gerichtetes Licht, das durch den
teilweise durchlässigen Spiegel austritt.\\
\\
Grundsätzlich müssen nun einige Bedingungen erfüllt sein, damit ein Laserbetrieb möglich ist. Zum einen muss die Verstärkung größer sein als die Verluste im Resonator. Verluste entstehen
zum Beispiel durch Absorption oder Streuung. Eine Verstärkung (erst unabhängig von Verlusten) ist gegeben, wenn der frequenzabhängige Absorptionskoeffizient $\alpha(\nu)$ negativ ist.
Für die Intensität entlang der Ausbreitungsrichtung $I(z)$ gilt dann:
\begin{equation*}
    I(z) = I_0 \exp\left(-\alpha(\nu)z\right).
\end{equation*}
Dadurch ergibt sich für die Verstärkung:
\begin{equation*}
    g(\nu) = \frac{I(z)}{I_0} = \exp\left(-\alpha(\nu)z\right).
\end{equation*}
Verluste im Resonator lassen sich mit dem Faktor $\exp(-\gamma)$ beschreiben. Für Hin- und Rückweg im Resonator der Länge $L$ gilt dann für die Verstärkung:
\begin{equation*}
    G = \exp(-2\alpha L - 2\gamma).
\end{equation*}
Die Bedingung für Laserbetrieb ist dann:
\begin{equation*}
    G > 1.
\end{equation*}


\newpage
