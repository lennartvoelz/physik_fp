\section{Experimental setup}
\label{sec:Aufbau}
\begin{figure}[H]
    \centering
    \includegraphics[width=0.5\textwidth]{aufbv47.pdf}
    \caption{Experimental setup for the measurement of the heat capacity of copper. \cite{v47}}
    \label{fig:a1}
\end{figure}
\noindent
\autoref{fig:a1} shows the experimental setup needed to measure the heat capacity of copper. It consists of a devar vessel, in which liquid nitrogen is filled to cool the recipient chamber with the sample and its shell. The sample and shell are both heated and are both connected to a resistor for temperature measurements of both components. The recipient is connected to a vacuum pump and a helium tank and is air tight in order to be vacuumated or filled with helium. The temperature is measured indirectly, by measuring the resistance of both the shell and the sample using two ohm meters. The time can be stopped using a digital stop watch.
\section{Execution}
\label{sec:Durchführung}
Before the measurement can take place, the recipient chamber has to be vacuumated. After the vacuum is established, the chamber is filled with helium, and the space between the recipient chamber and the devar vessel is filled with liquid nitrogen. Once a sample temperature of roughly $\SI{70}{K}$ is reached, the helium valve is closed and the helium is vacuumated out of the recipient vessel. The vacuum pump keeps running for the rest of the measurement. The current of the sample heater are then set to roughly $I_{Sa} = \SI{155}{mA}$ and the voltage is read of a volt meter. The shell current is adjusted manually to keep the shell and sample temperature equal. The timer is started and measures tabulated resistance ranges for the sample temperature resistor.

\newpage