\section{Introduction}
\label{sec:Zielsetzung}
Goal of this experiment is the measurement of the temperature dependency of the heat capacity $C_V$ at constant preassure and volume. 
The Debye temperature $\Theta_D$ will also be measured and compared to the theory value.

\section{Theory}
\label{sec:Theorie}
This section will give a brief overview of the theoretical framework needed to understand and interpret the following measurements. It will touch on the heat capacity $C_V$ in the context of solid state physics and explain the differences between three main theoretical models, describing the same phenomenon.
If not explicitly cited otherwise, this section will rely on the book "Festkörperphysik" \cite{rehab}.
\subsection{Classical theory of the heat capacity}
The molar heat capacity $C_V$, from now on referred to as heat capacity, is a thermodynamic variable describing the amount of heat $\Delta Q$ needed for a given temperature change of $\SI{1}{K}$. It is a material specific constant, which is different for different materials.
The heat capacity is given by the relation
\begin{equation}
    \label{eqn:t1}
    C_V = \left. \frac{\partial Q}{\partial T} \right|_{V, N} = \left. \frac{\partial U}{\partial T} \right|_{V, N}
\end{equation}
where $U$ is the solids inner energy. The heat capacity at constant preassure is given by
\begin{equation}
    C_p = \left. \frac{\partial Q}{\partial T} \right|_{p, N} = \left. \frac{\partial H}{\partial T} \right|_{p, N}
\end{equation}
where $H$ is the enthalpy of the system. The relation $C_V \leq C_p$ is true for any system, as the system at constant preassure has to expand when applying an amount of heat $\Delta Q$ to it, thus doing more work $\Delta W$ which increases the heat capacity.
$C_V$ and $C_p$ are fundamentally connected by the thermal expansion coefficient $\alpha$ and the compressibility $\kappa$ in the relation
\begin{equation}
    \label{eqn:t2}
    C_p - C_V = 9 \alpha^2 \kappa V_0 \, ,
\end{equation}
where $V_0$ is the molar volume of the solid.
To determine the heat capacity of a solid state body, we can approximate each atom in the solid as a harmonic oscillator, given by the Hamiltonian for a single atom in the solid 
\begin{equation}
    \mathcal{H} \left( \vec{p}, \vec{q} \right) = \frac{\vec{p}^2}{2m} + \frac{1}{2} k \vec{q}^2 \, .
\end{equation}
Using the equipartition theorem 
\begin{equation}
    \langle x_i \frac{\partial \mathcal{H}}{\partial x_i}\rangle = k_B T
\end{equation}
one can determine, that 
\begin{align}
    U = \langle \mathcal{H} \rangle &= \frac{1}{2} \sum_{i=1}^3 \langle \frac{p_i^2}{m} \rangle + \langle k q_i^2 \rangle \\
                                    &= \frac{1}{2} \sum_{i=1}^3 \langle p_i \frac{\partial \mathcal{H}}{\partial p_i} \rangle + \langle q_i \frac{\partial \mathcal{H}}{\partial q_i}\rangle \\
                                    &= 3 k_B T
\end{align}
is the inner energy of a single Atom in a solid state body.
By applying avogrados number $N_A$ and using \autoref{eqn:t1}, one can calculate the molar heat capacity of a solid state given by the doulong petite law
\begin{equation}
    C_V = 3 R T
\end{equation}
\subsection{Oscillation of the cristalline structure}
In order to understand the heat capacity of a solid state body, one has to first understand the oscillations of its cristalline structure, which will be introduced in this subsection.
The cristalline structure, when viewed from a purely mathematical perspective, becomes a set of vectors which are linear combinations of its basis vectors. This static view of a solid state body can be filled with motion when one imagines these vectors as springs, connecting each pair of neighbouring atoms in the lattice. The problem to be solved is then nothing more than the simple harmonic equations of motion every physicist has solved at least once in their live.
This approach neglects the electrons contribution to the overall problem. The introduced springs are nothing more than a quadratic approximation of the more complex electromagnetic potential each atom experiences. If we assume the electrons to be significantly faster than the atomic nuclei, individual electromagnetic contributions of the electrons to the electromagnetic potential one atom experiences in the lattice can be neglected as a static background.
This approach is known as the Born-Oppenheimer approximation and it allows us to approximate the electromagnetic potential in quadratic order, resulting in a harmonic potential which acts like a spring.
\begin{figure}[H]
    \centering
    \includegraphics[width=0.7\textwidth]{lattice.pdf}
    \caption{2D lattice of 2 atoms with equal couplings $f$ and different masses $M_1$ and $M_2$, \cite{rehab}.}
    \label{fig:t1}
\end{figure}
\noindent
The equations of motion for a 2D lattice of 2 Atoms with different masses $M_1$, $M_2$ and equal coupling $f$, see \autoref{fig:t1}, can then intuitively be written as 
\begin{align}
    M_1 \frac{\text{d}^2 u_n}{\text{dt}^2} &= f \left( v_n - u_n \right) + f \left( v_{n-1} - u_n \right) \\
    M_2 \frac{\text{d}^2 v_n}{\text{dt}^2} &= f \left( u_n - v_n \right) + f \left( u_{n+1} - v_n \right)
\end{align}
where $u_n$ and $v_n$ are the deflection function of each type of atom in a latice that can be seen in \autoref{fig:t1}.
Solving the equations for one row of this simple case, results in two solutions for the frequencies of the lattice oscillations, which can be seen in \autoref{fig:t2}.
\begin{figure}[H]
    \centering
    \includegraphics[width=0.7\textwidth]{dispersion.pdf}
    \caption{Dispersion relation for a 1D lattice of 2 atoms with different masses $M_1$, $M_2$ and equal coupling $f$, \cite{rehab}.}
    \label{fig:t2}
\end{figure}
\noindent
The upper branch of the dispersion relation in \autoref{fig:t2} is called the optical branch and the lower branch is called the accoustic branch. The optical branch is a result of the different masses $M_1$ and $M_2$ and is not present in a lattice with one atomic basis. There are $3N$ branches in a $3D$ lattice of $N$ different atoms, with $3N-3$ accoustic branches and $3$ optical branches, which are not dependent on the number of different atoms in the lattice.
The different branches of the dispersion relation of a solid state body are often referred to as phonons, wich are quantized lattice vibration. Phonons are Bosons and follow the Bose-Einstein distribution given by 
\begin{equation} 
    \label{eqn:t3}
    \langle n_\text{BE} \rangle = \frac{1}{\exp \left( \frac{\hbar \omega}{k_B T} \right) - 1} \, .
\end{equation}

\subsection{Einstein model}
\label{sec:t1}
The Einstein model approximates the heat capacity of a solid state body by assuming, that all phonons are of the same constant frequency $\omega_\text{E}$ called the Einstein frequency.
The inner energy of one mol of the system under this assumption is then given by 
\begin{equation}
    U = 3 N_A \hbar \omega_\text{E} \left( \langle n_\text{BE} \rangle + \frac{1}{2} \right) \, ,
\end{equation}
using \autoref{eqn:t1} the heat capacity of the system is given by
\begin{equation}
    C_V = 3 N_A k_B \left( \frac{\Theta_E}{T} \right)^2 \frac{\exp \left( \frac{\Theta_E}{T} \right)}{\left( \exp \left( \frac{\Theta_E}{T} \right) - 1 \right)^2} \, ,
\end{equation}
where $\Theta_E = \frac{\hbar \omega_E}{k_B}$ is the Einstein temperature.
The Einstein approximation dominates at high temperatures and is a good approximation for the optical phonons in a solid state body. Because this experiment uses copper, which has a single atom basis, the Einstein model is not a good approximation in this case as there will be no optical phonons in the lattice.
\subsection{Debye model}
The Debye model postulates a linear dispersion relation for all $3N$ phonon branches with an upper frequency $\omega_D$ the Debye-frequency. The dispersion relation for each branch is given by $\omega = v_s \cdot k$, where $v_s$ is the specific mode's speed of sound in the medium.
Because copper is a single basis atom, it has 3 accoustic vibrational modes of which two are transversal and one is longitudinal. This results in a density of states 
\begin{equation}
    Z \left(\omega\right) = \frac{L^3 \omega}{2 \pi^2 v_s^3}
\end{equation}
where $v_s$ is a function of the transversal and longitudinal speeds of sound in the medium $v_t$ and $v_l$, given by
\begin{equation}
    \label{eqn:t5}
    v_s^3 = \frac{3 v_l^3 v_t^3}{2v_l^3 + v_t^3} \, .
\end{equation}
Given the integral
\begin{equation}
U = \int_0^{\omega_D} Z\left(\omega\right) \langle n_{BE} \rangle \text{d}\omega \, ,
\end{equation}
The heat capacity can again easily be determined using \autoref{eqn:t1} resulting in an approximate solution for high and low temperatures to be
\begin{equation}
    C_V = 
    \begin{cases}
        \left(\frac{12 \pi^2}{5}\right) N k_B \left( \frac{T}{\Theta_D} \right)^3 &\quad \text{for } T \ll \Theta_D\\
        \frac{3}{2} k_B T &\quad \text{for } T \gg \Theta_D
    \end{cases}
    \, .
\end{equation}
$\Theta_D$ is the Debye temperature which is given by the expression
\begin{equation}
    \label{eqn:t4}
    \Theta_D = \frac{\hbar \omega_D}{k_B} = \frac{\hbar v_s}{k_B}\left( \frac{6 \pi^2 N}{V} \right)^{\frac{1}{3}} \, .
\end{equation}
The Debye approximation is more accurate for smaller temperatures, which is where accoustic phonons dominate the heat propagation. It is thus the approximation which will be used in this experiment, as it fits the single atom basis of the copper cristalline structure.



\newpage
