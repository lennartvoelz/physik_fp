\section{Structure of the experiment}
\label{sec:struc}
For the experiment a Sagnak interferometer is used, as depicted in \autoref{fig:sag}.
\begin{figure}[H]
    \centering
    \includegraphics[width=0.7\textwidth]{sagnak.pdf}
    \caption{illustration of the Sagnak interferometer \cite{sample}}
    \label{fig:sag}
\end{figure}
\noindent
The interferometer consists of 6 mirrors, of which M1 is used to calibrate the Laser beam directly without moving the Laser itself, M2 is used to split the beam into two parallel beams inside the interferometer, and MA to MC are mirrors of the interferometer itself and are used to align the beam correctly.
The interferometer is powered by a HeNe Laser, which has a wavelength of $\lambda = \SI{632.8}{\nano\meter}$ and a linear polarisation axis of fourtyfive degrees relative to the interferometer plane. Between M2 and MA the Polarising Beam Splitter Cube, short PBSC, can be seen, which splits the incoming beam into two beams, where the beam that is let through is polarised perpendicular to the interferometer plane and the beam that is reflected is polarised parallel to the interferometer plane. The two beams are running in opposite directions in the interferometer and are then recombined by the PBSC.
For the measurement of the beams intensity another PBSC which is tilted by fourtyfive degrees is used, which splits the beam to be measured by two seperate photodiodes, of which one experiences anihilation while the other experiences amplification. The two diodes are connected to a modern interferometry controller, which measures the difference of voltage to correct for stray light. For callibration a green Laser is used and callibration plates. For the polarisation measurements two polarisers are used and a vacuum pump configuration is installed for the measurement of the refractive index of air. 

\section{Conduction of the experiment} 
The first step of the experiment is the calibration of the interferometer. For this the HeNe is turned on and the mirror M1 is calibrated to point directly into the center of M2. After that a rigerous calibration process starts which is explained in detail in source \cite{sample}. Once the interferometer is calibrated, a polariser is installed infront of the PBSC and the glass plates between MC and the PBSC, in order to measure intensity maximum and minimum. The intensity maximum and minimum is then manually measured by a volt meter for 10 degree increments of the polariser 3 times. The polarisation angle is then set to the angle of highest contrast and intensity for the rest of the experiment. For the measurement of the refractive indices of glass the glass plates are installed in the interferometer and M2 is moved so that there are two seperate beams in the interferometer. The interferometer is readjusted and the modern interferometry controller is connected to an oscilloscope, which displays the differential voltage of the two photodiodes. The glass plates are then turned by a total of 10 degrees and the number of maxima are measured by the modern interferometry controller. This procedure is done several times, as well for the measurement of the refractive index of air which functions analogous, but this a vaccum chamber in one of the beams which is vaccumated and then slowly devacuumated. 

\label{sec:cond}
\newpage