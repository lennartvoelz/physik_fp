\section{Discussion}
\label{sec:Discussion}
In the following, the percentage deviations are calculated with
\begin{equation}\label{eq:1}
    \Delta = |\frac{exp - theo}{theo}|\cdot 100\%.
\end{equation}

\subsection{Contrast}
The maximum contrast is measured to be $C = \SI{0.897(0.024)}{}$ at an angle of $\phi = \SI{135}{\degree}$.
Expected is a maximum value at either $\SI{135}{\degree}$ or $\SI{45}{\degree}$ since the 
contrast should follow $C ~ |\sin(2x)|$. Therefore the experimental value is in agreement with the theoretical prediction.

\subsection{Refractive index of glass}
The refractive index of the glass plate is measured to be $n = \SI{1.51(0.04)}{}$. The theoretical value for the refractive index of glass is $n = 1.5168$ \cite{n_glass}. The percentage deviation is calculated to be $\Delta = \SI{0.44}{\percent}$.
This deviation is negligible and the theoretical value even lies in the uncertainty of the experimental value.

\subsection{Refractive index of air}
For the refractive index of air, the theoretical value is $n = 1.00027$ \cite{n_air}. This is again in
perfect agreement with on the one hand the experimental value and on the other hand the value from Lorentz-Lorenz law:
\begin{equation*}
    n_{LL} = n_{reg} = \SI{1.00027}{}.
\end{equation*}
\\
As one can see, there isn't much to discuss about the results.
Overall, it can be said that the measurement appears to be very accurate. 
This indicates good adjustment of the optical structure. 
In addition, the difference measurement and averaging over several measurement series appear 
to eliminate or limit statistical fluctuations.

\newpage