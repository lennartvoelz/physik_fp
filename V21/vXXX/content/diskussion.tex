\newpage
\section{Diskussion}
\label{sec:Diskussion}
Im Folgenden werden die prozentualen Abweichungen mit 
\begin{equation}\label{eq:1}
    \Delta = |\frac{exp - theo}{theo}|\cdot 100\%
\end{equation}
berechnet.
\subsection{Messung der Resonanzfeldstärke der Sweep Spule}
Die Messung der Resonanzfeldstärke der Sweep Spule in Abhängigkeit der Frequenz zeigt einen klaren linearen Zusammenhang, welcher durch einen Fit der Daten verdeutlicht wird und in Fitparametern resultiert die in den zu erwarteten Bereichen liegen. 
\subsection{Messung desLande-Faktors und des Kernspins}
Die Messung der Lande Faktoren und weisen mit $\Delta g_{F,1} = 4.2 \pm 3.5 \%$ und $\Delta g_{F,2} = 3 \pm 0.9 \%$ nur geringe Abweichungen von den Theoriewerten auf. Die Werte für die Kernspins haben ebenfalls mit 
$\Delta I_{1} = 2.1 \pm 1.7 \%$ und $\Delta I_{2} = 0.81 \pm 0.25 \%$ nur sehr geringe Abweichungen von den Theoriewerten.
\subsection{Messung des Isotopenverhältnisses und abschätzung des quadratischen Zeeman-Effekts}
Bei beiden Teilen der Auswertung gab es keine auffälligkeiten in der Bestimmung der Werte. Die Anteile der beiden Isotope entsprechen den Erwartungen und die Übergangsenergien aus \autoref{sec:Zeeman} sind ebenfalls in den zu erwartenden Bereichen.
\newpage