\section{Diskussion}
\label{sec:Diskussion}
Im Folgenden werden die prozentualen Abweichungen mit 
\begin{equation}\label{eq:1}
    \Delta = |\frac{exp - theo}{theo}|\cdot 100\%
\end{equation}
berechnet.
\subsection{Messung der Resonanzfeldstärke der Sweep Spule}
Auffällig bei der Durchführung der Messreihe ist, dass in der Anleitung ein Wert von $0.1 \frac{\mathrm{A}}{\text{Umdrehung}}$ des Potentiometers gegeben ist. Dieser scheint aber falsch zu sein, da
sich der Versatz der Peaks gegeben durch $U_{\text{ho}}$ der Horizontalen Spule nach der Umrechnung in Stromstärke in einer viel zu kleinen Größenordnung befindet. Diese war so klein, dass die Spannung der Horizontalen Spule kaum auswirkung auf die Messwerte hatte und kein klar linearer zusammenhang zu erkennen war.
Bei erhöhen des Beitrages der Horizontalen Spule um einen Faktor $10$ kam es zu einem erkennbaren linearen zusammenhang der Messwerte, jedoch wichen die Ergebnisse um einen Faktor $10$ von der Theorie ab. Daher wurde davon ausgegangen, dass die Angabe von $0.1 \frac{\mathrm{A}}{\text{Umdrehung}}$ inkorrekt ist und es wurde mit einem Faktor $0.01 \frac{\mathrm{A}}{\text{Umdrehung}}$ weiter gerechnet.
\subsection{Messung desLande-Faktors und des Kernspins}
Trotz der schwierigkeiten bei mit den gegebenen Umrechnungsfaktoren sind die Ergebnisse der Messung von guter Qualität und weisen mit $\Delta g_{F,1} = 2 \pm 6 \%$ und $\Delta g_{F,2} = 3 \pm 1.1 \%$ nur geringe Abweichungen von den Theoriewerten auf. Die Werte für die Kernspins weisen dementsprechend mit 
$\Delta I_{1} = 1.2 \pm 2.9 \%$ und $\Delta I_{2} = 0.8 \pm 0.3 \%$ ebenfalls nur sehr geringe Abweichungen auf.
\newpage