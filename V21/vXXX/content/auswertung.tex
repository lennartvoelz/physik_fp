\section{Auswertung}
\label{sec:Auswertung}
Die in \autoref{sec:Auswertung} gezeigten Grafiken und Rechnungen sind mithilfe der Python-Bibliotheken Matplotlib \cite{matplotlib}, Scipy \cite{scipy} und Numpy \cite{numpy}
erstellt worden. Die Fehlerrechnung wird mithilfe der Python-Bibliothek Uncertainties \cite{uncertainties} durchgeführt.
%Tabellen nebeneinander
\begin{table}[H]
  \centering
  \caption{}
  \begin{tabular}{S[table-format=3] S[table-format=2(2)]}
      \toprule
      {$t\,/\symup{s}$} & {$N$} \\
      \midrule

      \bottomrule
  \end{tabular}
  \begin{tabular}{S[table-format=3] S[table-format=2(2)]}
      \toprule
      {$t\,/\symup{s}$} & {$N$} \\
      \midrule
      
      \bottomrule
  \end{tabular}
  \label{tab:}
\end{table}
%Plots und Bilder
%\begin{figure}[H]
%  \includegraphics[width=\linewidth]{plots/.pdf}
%  \caption{}
%  \label{fig:}
%\end{figure}
\subsection{Messung der Resonanzfeldstärke der Sweep Spule}
\label{sec:Resonanzfeldstärke}
Für die Messung der Resonanzfeldstärke der Sweep Spule wurde die Apperatur manuell so eingestellt, dass parallel zu der Nord Süd Achse des Erdmagnetfeldes steht. Zusätzlich musste die Vertikalkomponente 
des Erdmagnetfeldes kompensiert werden, was durch einstellen des Stromes in der Vertikalspule erreicht wurde. Der Strom in dieser Spule wurde auf $I_{\text{vertikal}} = \SI{0.0244}{\ampere}$ gestellt  was nach Gleichung $cite(Helmholtz)$ einer magnetischen Feldstärke von $B_{\text{vertikal}} \approx \SI{21.4}{\micro\tesla}$ entspricht.
Für die Messung der Resonanzfeldstärke 
\newpage